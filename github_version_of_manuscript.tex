\documentclass[11pt, letterpaper]{amsart}
\usepackage[left=1in,right=1in,bottom=1in,top=1in]{geometry}
\usepackage{amsfonts}
\usepackage{amsmath, amssymb}
\usepackage{graphicx}
\usepackage[font=small,labelfont=bf]{caption}
\usepackage{epstopdf}
\usepackage[usenames,dvipsnames]{xcolor}
\usepackage[pdfpagelabels,hyperindex,colorlinks=true,linkcolor=blue,urlcolor=magenta,citecolor=green]{hyperref}
\usepackage{amsthm}
\usepackage{float}
\usepackage{pgfplots}
\usepackage{listings}
\usepackage{longtable}
\usepackage{mathrsfs}

\hypersetup{
pdftitle={Another Power Identity involving Binomial Theorem and Faulhaber's formula},
pdfsubject={Mathematics, Number Theory, Combinatorics},
pdfauthor={Petro Kolosov},
pdfkeywords={Faulhaber's formula, Faulhaber's theorem, Binomial Theorem, Binomial coefficient, Binomial distribution,
Binomial identities, Power Sums, Finite differences}
}

\newcommand{\overbar}[1]{\mkern 1.5mu\overline{\mkern-1.5mu#1\mkern-1.5mu}\mkern 1.5mu}
\DeclareRobustCommand{\stirling}{\genfrac\{\}{0pt}{}}
\newcommand{\urlNewWindow}[1]{\href[pdfnewwindow=true]{#1}{\nolinkurl{#1}}}
\newcommand\ddfrac[2]{\frac{\displaystyle #1}{\displaystyle #2}}
\newtheorem{thm}{Theorem}[section]
\newtheorem{cor}[thm]{Corollary}
\newtheorem{prop}[thm]{Proposition}
\newtheorem{lem}[thm]{Lemma}
\newtheorem{conj}[thm]{Conjecture}
\newtheorem{quest}[thm]{Question}
\newtheorem{ppty}[thm]{Property}
\newtheorem{ppties}[thm]{Properties}
\newtheorem{claim}[thm]{Claim}

\theoremstyle{definition}
\newtheorem{defn}[thm]{Definition}
\newtheorem{defns}[thm]{Definitions}
\newtheorem{con}[thm]{Construction}
\newtheorem{exmp}[thm]{Example}
\newtheorem{exmps}[thm]{Examples}
\newtheorem{notn}[thm]{Notation}
\newtheorem{notns}[thm]{Notations}
\newtheorem{addm}[thm]{Addendum}
\newtheorem{exer}[thm]{Exercise}
\newtheorem{limit}[thm]{Limitation}

\theoremstyle{remark}
\newtheorem{rem}[thm]{Remark}
\newtheorem{rems}[thm]{Remarks}
\newtheorem{warn}[thm]{Warning}
\newtheorem{sch}[thm]{Scholium}

\definecolor{energy}{RGB}{114,0,172}
\definecolor{freq}{RGB}{45,177,93}
\definecolor{spin}{RGB}{251,0,29}
\definecolor{signal}{RGB}{203,23,206}
\definecolor{circle}{RGB}{217,86,16}
\definecolor{average}{RGB}{203,23,206}
\newcommand{\K}{\operatornamewithlimits{K}}
\colorlet{shadecolor}{gray!20}
\pgfplotsset{compat=1.9}
\def\N{10}
\def\M{4}
\usepgflibrary{fpu}


\makeatletter
\let\c@equation\c@thm
\raggedbottom
\makeatother
\numberwithin{equation}{section}
%--------Meta Data: Fill in your info------
%&#1043;&#1086;&#1089;&#1087;&#1086;&#1076;&#1080;, &#1048;&#1080;&#1089;&#1091;&#1089;&#1077; &#1061;&#1088;&#1080;&#1089;&#1090;&#1077;, &#1057;&#1099;&#1085;&#1077; &#1041;&#1086;&#1078;&#1080;&#1081;, &#1084;&#1086;&#1083;&#1080;&#1090;&#1074;&#1072;&#1084;&#1080; &#1055;&#1088;&#1077;&#1095;&#1080;&#1089;&#1090;&#1099;&#1103; &#1058;&#1074;&#1086;&#1077;&#1103; &#1052;&#1072;&#1090;&#1077;&#1088;&#1080; &#1080; &#1042;&#1089;&#1077;&#1093; &#1057;&#1074;&#1103;&#1090;&#1099;&#1093; &#1058;&#1074;&#1086;&#1080;&#1093;, &#1087;&#1086;&#1084;&#1080;&#1083;&#1091;&#1081; &#1085;&#1072;&#1089;. &#1040;&#1084;&#1080;&#1085;&#1100;.
\author[Petro Kolosov]{Petro Kolosov}
\email{kolosovp94@gmail.com}
\keywords{Faulhaber's formula, Faulhaber's theorem, Binomial Theorem, Binomial coefficient, Binomial distribution,
Binomial identities, Power Sums, Finite differences}
\urladdr{https://kolosovpetro.github.io}
\subjclass[2010]{11C08 (primary), 41A10 (secondary)}
\title{Another Power Identity involving Binomial Theorem and Faulhaber's formula}
\begin{document}
\begin{abstract}
In this paper, by means of Binomial theorem and Faulhaber's formula we derive and prove the following identity
between odd power and $m$-order polynomials
$$T\strut^{2s+1}=\sum_{r=0}^{s}\sum_{\kappa=1}^{2s-r+1} (-1)\strut^{2s-r} \mathscr{L}_{s,r}(\kappa) \cdot T\strut^{\kappa+r}.$$
\end{abstract}
\maketitle
\tableofcontents
\section{Introduction and Main results}\label{intro}
Define the $m$-order polynomials $P_m(\ell,T)$ in $T$
\begin{equation}\label{f1}
P_m(\ell,T) := \sum_{k=1}^{\ell}\sum_{j=0}^m A_{m,j}k\strut^j(T-k)\strut^j=\sum_{k=0}^{m}(-1)\strut^{m-k}U_m(\ell, k)\cdot T\strut^k,
\end{equation}
where $\ell\in\mathbb{N}, \ m\in\mathbb{N}$, $T\in\mathbb{R}$ and $U_m(\ell, k), \ A_{m,j}$ are coefficients defined as follows
\begin{defn}\label{gen_13}
\textit{(Definition of $A_{m,j}$ coefficients.)}
\begin{equation*}
A_{m,j}:=
\begin{cases}
0, & \mathrm{if } \ j<0 \ \mathrm{or } \ j>m, \\
(2j+1)\binom{2j}{j} \sum_{d=2j+1}^{m} A_{m,d} \binom{d}{2j+1} \frac{(-1)\strut^{d-1}}{d-j} B_{2d-2j}, & \mathrm{if } \ 0 \leq j < m, \\
(2j+1)\binom{2j}{j}, & \mathrm{if } \ j=m. \\
\end{cases}
\end{equation*}
\end{defn}

\begin{defn}
\textit{(Definition of $U_m(\ell, k)$ coefficients.)}
\begin{equation*}
U_m(\ell, k) := (-1)\strut^m \sum_{k=a}^{\ell}\sum_{j=t}^m (-1)\strut^j\binom{j}{t}A_{m,j}k\strut^{2j-t}.
\end{equation*}
\end{defn}
From this formula it may be not immediately clear why $U_m(\ell,t)$ represent polynomials in $\ell$. However, this can be seen if we change the summation order again and use Faulhaber's formula to obtain:
\begin{equation}\label{maxal1}
U_m(\ell,t) = (-1)^m \sum_{j=t}^m \binom{j}{t}A_{m,j} \frac{(-1)^j}{2j-t+1}\sum_{l=0}^{2j-t} \binom{2j-t+1}{l}B_{l}\ell^{2j-t+1-l}.
\end{equation}
Introducing $\kappa=2j-t+1-l$ to (\hyperref[maxal1]{2.31}) we further get the formula:
\begin{equation}
U_m(\ell,t) = (-1)^m \sum_{\kappa=1}^{2m-t+1} \ell^\kappa \sum_{j=t}^m \binom{j}{t}A_{m,j} \frac{(-1)^j}{2j-t+1}\binom{2j-t+1}{\kappa}B_{2j-t+1-\kappa},
\end{equation}
which allows easily compute the coefficient of $\ell^{\kappa}$ in $U_m(\ell,t)$ for each $\kappa$. In above formulae we assume that $B_1=+\frac12$. We define the coefficient for each $\ell^{\kappa}$ in $U_m(\ell,t)$ as follows
\begin{defn} \textit{(Definition of $\mathscr{L}_{m,r}(\kappa)$ coefficients.)}
\begin{equation*}
\mathscr{L}_{m,r}(\kappa)=\sum_{j=r}^{m} \binom{j}{r}A_{m,j} \frac{(-1)^j}{2j-r+1}\binom{2j-r+1}{\kappa}B_{2j-r+1-\kappa}
\end{equation*}
\end{defn}
Let's define the polynomial $\sum\nolimits_{0\leq j \leq m} A_{m,j}k\strut^j(T-k)\strut^j$ that is the part of (\hyperref[f1]{1.1}) as follows
\begin{defn} \label{symmetry_1} \textit{(Definition of $g_m(k,T)$ polynomial.)}
\begin{equation*}
g_m(k,T) := \sum_{j=0}^{m} A_{m,j}k\strut^j(T-k)\strut^j.
\end{equation*}
\end{defn}
\begin{ppty}\label{symmetry_2} (Symmetry of $g_m(k,T)$.)
The values of (\hyperref[symmetry_2]{1.3}) is always being distributed symmetrically over given value of $\tfrac{T}{2}$, i.e
\begin{equation*}
g_m(T,k)=g_m(T,T-k).
\end{equation*}
\end{ppty}

The partial case of (\hyperref[f1]{1.1}) for $\ell=T\in\mathbb{N}$ gives
\begin{equation}\label{main_identity}
\boxed
{
T^{2m+1}=\sum_{k=1}^{T}\sum_{j=0}^m A_{m,j}k\strut^j(T-k)\strut^j=\sum_{k=0}^{m}(-1)^{m-k}U_m(\ell, k)\cdot T\strut^k=P_m(T,T)
}
\end{equation}
By rewriting the right-hand side of equation (1.9) in terms of $\mathscr{L}_{m,r}(\kappa)$ coefficients for every $s\geq 0$ we get
\begin{equation}
T\strut^{2s+1}=\sum_{r=0}^{s}\sum_{\kappa=1}^{2s-r+1} (-1)\strut^{2s-r} \mathscr{L}_{s,r}(\kappa) \cdot T\strut^{\kappa+r}.
\end{equation}
\section{Derivations and examples}
\subsection{Example for \texorpdfstring{$m=1$}{m=1}}
We start our discussion concerning the polynomials (\hyperref[f1]{1.1}) from the derivation of the partial case $P_m(T,T)=T^{2m+1}$ of the polynomial $P_m(\ell,T)$ for $m=1$. Consider the Faulhaber's identities \cite{1} for odd powers $n^{2m-1}, \ m\in\mathbb{N}$
\begin{equation}\label{knuth1}
n^{2m-1}=\begin{cases}
n^1 = \binom{n}{1}, \ &\mathrm{if} \ m=1;\\
n^3 = 6\binom{n+1}{3}+\binom{n}{1}, \ &\mathrm{if} \ m=2;\\
n^5 = 120\binom{n+2}{5}+30\binom{n+1}{3}+\binom{n}{1}, \ &\mathrm{if} \ m=3;\\
\vdots\\
n^{2m-1} = \sum\limits_{1\leq k\leq m}(2k-1)!T(2m,2k)\binom{n+k-1}{2k-1}, \ &\mathrm{if} \ m\in\mathbb{N};
\end{cases}
\end{equation}
The coefficients $(2k-1)!T(2m,2k)$ in the Faulhaber's identities (\hyperref[knuth1]{2.1}) can be calculated using following formula
\begin{equation}\label{knuth2}
(2k-1)!T(2n,2k)=\frac{1}{r}\sum_{j=0}^{r}(-1)^j\binom{2r}{j}(r-j)^{2n},
\end{equation}
where $r=n-k+1$ and $T(2n,2k)$ is central factorial number, see \cite{22}. The formula (\hyperref[knuth2]{2.2}) was derived by Peter Luschny in \cite{27}. The $r$-order forward finite difference of odd power $n^{2m-1}$ could be reached introducing the $r\geq 1$ to the lower index of binomial coefficient of (\hyperref[knuth1]{2.1}). For instance,
\begin{equation}\label{forward_diff_identity}
\Delta^r n^{2m-1}=
\begin{cases}
\Delta n^1 = \binom{n}{1-r}=\binom{n}{0}, \ &\mathrm{if} \ m=1, \ r=1;\\
\Delta n^3 = 6\binom{n+1}{3-r}+\binom{n}{1-r}=6\binom{n+1}{2}+\binom{n}{0}, \ &\mathrm{if} \ m=2, \ r=1;\\
\Delta n^5 = 120\binom{n+2}{5-r}+30\binom{n+1}{3-r}+\binom{n}{1-r}=120\binom{n+2}{4}+30\binom{n+1}{2}+\binom{n}{0}, \ &\mathrm{if} \ m=3, \ r=1;\\
\vdots\\
\Delta n^{2m-1} =\sum\limits_{1\leq k\leq m}(2k-1)!T(2m,2k)\binom{n+k-1}{2k-2}, \ &\mathrm{if} \ m\in\mathbb{N}, \ r=1;
\end{cases}
\end{equation}
By the identity $\Delta f(x-1)=\nabla f(x), \ h=1$, backward differences could be reached as well,
\begin{equation}\label{backward_diff_identity}
\nabla^r n^{2m-1}=
\begin{cases}
\nabla n^1 = \binom{n-1}{1-r}                                      = \binom{n-1}{0},               \ &\mathrm{if} \ m=1, \ r=1;\\
\nabla n^3 = 6\binom{n}{3-r}+\binom{n-1}{1-r}                      = 6\binom{n}{2}+\binom{n-1}{0}, \ &\mathrm{if} \ m=2, \ r=1;\\
\nabla n^5 = 120\binom{n+1}{5-r}+30\binom{n}{3-r}+\binom{n-1}{1-r} = 120\binom{n+1}{4}+30\binom{n}{2}+\binom{n-1}{0}, \ &\mathrm{if} \ m=3, \ r=1;\\
\vdots\\
\nabla n^{2m-1} = \sum\limits_{1\leq k\leq m}(2k-1)!T(2m,2k)\binom{n+k-2}{2k-2}, \ &\mathrm{if} \ m\in\mathbb{N}, \ r=1;
\end{cases}
\end{equation}
Now, let's derive the partial case $P_m(T,T)=T^{2m+1}$ of the polynomial (\hyperref[f1]{1.1}) for $m=1$ by means of Faulhaber's identities in forward and backward differences of cubes, see (\hyperref[forward_diff_identity]{2.3}), (\hyperref[backward_diff_identity]{2.4})
\begin{equation}\label{f6}
\begin{split}
T^3
=&\sum\limits_{k=0}^{T-1}\Delta T^3(k)=\sum\limits_{k=0}^{T-1}6\binom{k+1}{2}+\binom{k}{0}\\
=&\sum\limits_{k=1}^{T}\nabla T^3(k)=\sum\limits_{k=1}^{T}6\binom{k}{2}+\binom{k-1}{0}.
\end{split}
\end{equation}
Replacing the binomial coefficient $\tbinom{n}{2}$ in (\hyperref[f6]{2.5}) by the r.h.s of the identity $\tbinom{n}{2}=1+2+\cdots+(n-1)$, we get
\begin{equation}\label{f7}
T^3=(1+6\cdot0)+(1+6\cdot0+6\cdot1)+\cdots+(1+6\cdot0+\cdots+6\cdot(T-1))
\end{equation}
Note that the identity $\tbinom{n}{2}=1+2+\cdots+(n-1)$ can be generalized to the "Hockey Stick Pattern" in terms of binomial coefficients,
\begin{equation}\label{hockey_stick}
\binom{n}{k}=\sum_{j=k-1}^{n}\binom{j}{k-1}.
\end{equation}
see \urlNewWindow{https://www.cut-the-knot.org/arithmetic/combinatorics/PascalTriangleProperties.shtml}.
Factorising the expression (\hyperref[f7]{2.6}) we get
\begin{equation}\label{f8}
T^3=T+(T-0)\cdot6\cdot0+(T-1)\cdot6\cdot1+\cdots+(T-(T-1))\cdot6\cdot(T-1)
\end{equation}
Rewrite the expression (\hyperref[f8]{2.7}) in compact form, we have
\begin{equation}\label{cubbe}
T^3=T+\sum\limits_{k=1}^{T}6k(T-k) = \sum\limits_{k=1}^{T}6k(T-k)+1=P_1(T,T),
\end{equation}
see (\hyperref[f1]{1.1}) for $P_1(T,T)$.
The corresponding coefficients $A_{m,j}, \ 0\leq m \leq 1$ in the (\hyperref[cubbe]{2.8}) are: $A_{1,0}=1, \ A_{1,1}=6$.
Although, the expression (\hyperref[cubbe]{2.8}) is derived, essentially, from identity in finite differences of cubes, the finite differences $\Delta T^3(k)\neq6k(T-k)+1$ and $\nabla T^3(k)\neq6k(T-k)+1$. Over the manuscript we review the $P_m(\ell, T)$ in sense of backward difference $\nabla T^{2m+1}, \ m\in\mathbb{N}_1$, i.e we use the summation limits over $k$ as $1\leq k\leq \ell$ in (\hyperref[f1]{1.1}). To show that symmetry (\hyperref[symmetry_2]{1.4}) holds for (\hyperref[cubbe]{2.8}), let's construct a triangular array filled by the values of $g_1(T,k)=6k(T-k)+1$, where $T$ is row and $k$ shows the $k$-th term of the $T-$th row, see (\hyperref[symmetry_1]{1.3}) for definition of $g_m(T,k)$
\begin{table}[H]
\centering
\begin{tabular}{lrrrrrrrrrrr}
$T=0$&  &    &    &    &    &  1 &    &    &    &    &   \\\noalign{\smallskip\smallskip}
$T=1$&  &    &    &    &  1 &    &  1 &    &    &    &   \\\noalign{\smallskip\smallskip}
$T=2$&  &    &    &  1 &    &  7 &    &  1 &    &    &   \\\noalign{\smallskip\smallskip}
$T=3$&  &    &  1 &    &  13 &    &  13 &    &  1 &    &   \\\noalign{\smallskip\smallskip}
$T=4$&  &  1 &    &  19 &    &  25 &    &  19 &    &  1 &   \\\noalign{\smallskip\smallskip}
$T=5$&1 &    &  25 &    & 37 &    & 37 &    &  25 &    & 1 \\\noalign{\smallskip\smallskip}
\end{tabular}
\caption{Triangle generated by $g_1(T,k)=6k(T-k)+1, \ 0\leq T\leq5, \ 0\leq k\leq T$,
sequence \urlNewWindow{https://oeis.org/A287326} in OEIS, \cite{3}.} \label{fig_1}
\end{table}
The sum of the $T$-th row terms of \hyperref[fig_1]{Table (1)} starting from $k=1$ gives the partial case $P_1(T,T)=T^3$ of (\hyperref[f1]{1.1}). Binomial distribution of the row terms of \hyperref[fig_1]{Table (1)} can be easily proven by reviewing of its generating function $g_1(T,k)=\sum_{j=0}^1 A_{1,j}k^j(T-k)^j=6k(T-k)+1=6kT-6k^2+1$, which is parabolic for every given $T=const\in\mathbb{N}$, and, therefore, is symmetrical over $\tfrac{T}{2}$. Below we show a few properties of the $g_1(T,k)$
\begin{enumerate}
  \item Finite difference property, $\Delta(T^3)=g_1\left(\frac{T^2+T+2}{2}, 1\right)$, where $\frac{T^2+T+2}{2}$ is the $T$-th Central polygonal number, see \cite{29}.
  \item Binomial transform of the second order of $g_1(T,k)$: $g_1(T+1,k)=2g_1(T,k)-g_1(T-1,k)$.
  \item Symmetry, $g_1(T,k)=g_1(T, T-k), \ 0\leq k\leq T$.
  \item The sum $\sum_k g_1(k,1)$ gives octagonal numbers, or so-called, star numbers, \urlNewWindow{https://oeis.org/A000567}.
\end{enumerate}
Below we show initial ten polynomials $P_m(\ell,T)=\sum_{k=0}^{m}(-1)^{m-k}U_m(\ell,k)\cdot T^k$ for $m=1$:
\begin{equation}\label{u_1}
\begin{split}
P_1(\ell, T) &= \sum_{k=1}^{\ell}\sum_{j=0}^1 A_{1,j}k^j(T-k)^j =
\sum_{k=1}^{\ell}6k(T-k)+1=\sum_{k=0}^{1}(-1)^{1-k}U_1(\ell,k)\cdot T^k \\
&=-3 \ell^2 - 2 \ell^3 + 3 \ell T + 3 \ell^2 T\\
&=\begin{cases}
\ell=1 : -5 + 6 T &= -U_1(1,0)\cdot T^0+U_1(1,1)\cdot T^1=P_1(1,T)\\
\ell=2 : -28 + 18 T &= -U_1(2,0)\cdot T^0+U_1(2,1)\cdot T^1=P_1(2,T)\\
\ell=3 : -81 + 36 T &= -U_1(3,0)\cdot T^0+U_1(3,1)\cdot T^1=P_1(3,T)\\
\ell=4 : -176 + 60 T &= -U_1(4,0)\cdot T^0+U_1(4,1)\cdot T^1=P_1(4,T)\\
\ell=5 : -325 + 90 T &= -U_1(5,0)\cdot T^0+U_1(5,1)\cdot T^1=P_1(5,T)\\
\ell=6 : -540 + 126 T &= -U_1(6,0)\cdot T^0+U_1(6,1)\cdot T^1=P_1(6,T)\\
\ell=7 : -833 + 168 T &= -U_1(7,0)\cdot T^0+U_1(7,1)\cdot T^1=P_1(7,T)\\
\ell=8 : -1216 + 216 T &= -U_1(8,0)\cdot T^0+U_1(8,1)\cdot T^1=P_1(8,T)\\
\ell=9 : -1701 + 270 T &= -U_1(9,0)\cdot T^0+U_1(9,1)\cdot T^1=P_1(9,T)\\
\ell=10: -2300 + 330 T &= -U_1(10,0)\cdot T^0+U_1(10,1)\cdot T^1=P_1(10,T)
\end{cases}\\
&=T^3 \ \mathrm{as} \ \ell = T\in\mathbb{N}.
\end{split}
\end{equation}
The coefficients $U_1(\ell,k), \ 0\leq k\leq 1$ in (\hyperref[u_1]{2.9}) are terms of the sequence \urlNewWindow{https://oeis.org/A320047}, \cite{28}. The following figure shows graphically the intersections of the binomials (\hyperref[u_1]{2.9}) with monomial $T^3$ for $\ell\leq3$:
\begin{figure}[H]
\centering
\includegraphics[width=400px, keepaspectratio]{figure_n_3.eps}
\captionof{figure}{Intersections of binomials $P_1(\ell_1=1,T), \ P_1(\ell_2=2,T), \ P_1(\ell_3=3,T)$ and monomial $T^3$ in points $\ell_1=1, \ \ell_2=2, \ \ell_3=3$ according to the identity (\hyperref[main_identity]{1.2}).}
\end{figure}
\subsection{Example for \texorpdfstring{$m=2$}{m=2}}
Consider the case $m=2$ in polynomial (\hyperref[f1]{1.1}), we have $P_2(\ell,T)=\sum_{k=1}^{\ell}\sum_{j=0}^2 A_{2,j}k^j(T-k)^j=T^5$, as $\ell=T\in\mathbb{N}$. Now, we have to determine the coefficients $A_{2,0}, \ A_{2,1}, \ A_{2,2}$ in the polynomial $P_2(\ell,T)$. We rewrite the polynomial $P_2(\ell,T)$ in extended view as follows
\begin{equation}\label{alb_3_1}
\begin{split}
&\sum_{k=1}^{\ell}\sum_{j=0}^2 A_{2,j}k^j(T-k)^j\\
=A_{2,2}&\left(\sum\limits_{1\leq k \leq \ell}k^2(T-k)^2\right)+A_{2,1}\left(\sum\limits_{1\leq k \leq \ell}k^1(T-k)^1\right)+A_{2,0}\left(\sum\limits_{1\leq k \leq \ell}1\right)\\
=A_{2,2}&\left(\sum\limits_{1\leq k \leq \ell} k^2(T^2-2Tk+k^2)\right)+A_{2,1}\left(\sum\limits_{1\leq k \leq \ell} kT-k^2\right) + A_{2,0}\left(\sum\limits_{1\leq k \leq \ell}1\right)\\
=A_{2,2}&\left(\sum\limits_{1\leq k \leq \ell} k^2T^2-2Tk^3+k^4\right)+A_{2,1}\left(\sum\limits_{1\leq k \leq \ell} kT-k^2\right) + A_{2,0}\left(\sum\limits_{1\leq k \leq \ell}1\right) \\
=A_{2,2}&T^2\left(\sum\limits_{1\leq k \leq \ell}k^2\right)-2A_{2,2}T\left(\sum\limits_{1\leq k \leq \ell}k^3\right)+A_{2,2}\left(\sum\limits_{1\leq k \leq \ell}k^4\right)+A_{2,1}T\left(\sum\limits_{1\leq k \leq \ell} k\right) \\
-A_{2,1}&\left(\sum\limits_{1\leq k \leq \ell}k^2\right)+A_{2,0}\left(\sum\limits_{1\leq k \leq \ell}1\right)=T^5, \ \mathrm{as} \ \ell=T\in\mathbb{N}.
\end{split}
\end{equation}
By Binomial Theorem and Faulhaber's formula the expression (\hyperref[alb_3_1]{2.10}) can be found for every $m\in\mathbb{N}$ and $\ell=T\in\mathbb{N}$ as follows
\begin{equation*}
\begin{split}
T^{2m+1}
&=\sum_{j=0}^{m}\sum_{s=0}^{j}\sum_{k=1}^{T}(-1)^s \binom{j}{s} A_{m,j} k^{j+s} T^{j-s}=\sum_{j=0}^{m}\sum_{s=0}^{j} T^{j-s}(-1)^s \binom{j}{s} A_{m,j}\left(\sum_{k=1}^{T} k^{j+s}\right)\\
&=\sum_{j=0}^{m}\sum_{s=0}^{j} T^{j-s}(-1)^s \binom{j}{s} A_{m,j}\left[\sum_{r=0}^{j+s}\frac{(-1)^r}{j+s+1}\binom{j+s+1}{j}B_{j}T^{j+s+1-r}\right]\\
&=\sum_{j=0}^{m}\sum_{s=0}^{j}\sum_{r=0}^{j+s}\binom{j}{s} A_{m,j}\frac{(-1)^{r+s}}{j+s+1}\binom{j+s+1}{j}B_{j}T^{2j+1-r}\\
&=\sum_{j=0}^{m}\sum_{s=0}^{j}\binom{j}{s} A_{m,j}\frac{(-1)^s}{(j+s+1)(T+1)}\binom{j+s+1}{j}B_{j}[(-1)^{j+2s}T^{j-s+1}+T^{2j+1}]\\
&=\sum_{j=0}^{m}\sum_{s=0}^{j}\binom{j}{s} A_{m,j}\frac{(-1)^s}{(j+s+1)(T+1)}\binom{j+s+1}{j}B_{j}(-1)^{j+2s}T^{j-s+1}\\
&+\sum_{j=0}^{m}\sum_{s=0}^{j}\binom{j}{s} A_{m,j}\frac{(-1)^s}{(j+s+1)(T+1)}\binom{j+s+1}{j}B_{j}T^{2j+1}
\end{split}
\end{equation*}
Note that the part $A_{2,0}\sum\nolimits_{1\leq k \leq \ell}k^0(T-k)^0$ of (\hyperref[alb_3_1]{2.10}) gives an indeterminate form for $k=T$ as the $A_{2,0}(T-T)^0k^0$ contains the term $(T-T)^0=0^0$. Some textbooks leave the quantity $0^0$ undefined, because the functions $x^0$ and $0^x$ have different limiting values when $x$ decreases to $0$. For our purposes we will use the convention:
$$\forall x: \ x^0 = 1,$$
as it is a common agreement, see \cite{16}.
By the Faulhaber's formula, the sums of successive powers in (\hyperref[alb_3_1]{2.10}) are following
\begin{equation}\label{alb_3_1_1}
\begin{split}
&\sum\limits_{1\leq k \leq \ell} k = \frac{\ell^2+\ell}{2}, \\
&\sum\limits_{1\leq k \leq \ell} k^2 = \frac{2\ell^3+3\ell^2+\ell}{6}, \\
&\sum\limits_{1\leq k \leq \ell} k^3 = \frac{\ell^4+2\ell^3+\ell^2}{4}, \\
&\sum\limits_{1\leq k \leq \ell} k^4 = \frac{6\ell^5+15\ell^4+10\ell^3-\ell}{30}.
\end{split}
\end{equation}
Next, we substitute the identities (\hyperref[alb_3_1_1]{2.11}), into (\hyperref[alb_3_1]{2.10}) respectively
\begin{equation}\label{alb_3_1_2}
\begin{split}
A_{2,2}T^2\left(\frac{2\ell^3+3\ell^2+\ell}{6}\right)
-&2A_{2,2}T\left(\frac{\ell^4+2\ell^3+\ell^2}{4}\right)+A_{2,2}\left(\frac{6\ell^5+15\ell^4+10\ell^3-\ell}{30}\right)\\
+&A_{2,1}T\left(\frac{\ell^2+\ell}{2}\right)-A_{2,1}\left(\frac{2\ell^3+3\ell^2+\ell}{6}\right)+A_{2,0}\ell.
\end{split}
\end{equation}
Factorising the expression (\hyperref[alb_3_1_2]{2.12}) and rewriting it under common divisor with set $\ell=T\in\mathbb{N}$, we get
\begin{equation}\label{alb_3}
\frac{A_{2,2}T^5-A_{2,2}T+30A_{2,0}}{30}+A_{2,1}\frac{T^3-T}{6}=T^5.
\end{equation}
In order to satisfy (\hyperref[alb_3]{2.13}) for every $T\in\mathbb{N}$, the coefficients $A_{2,0}, \ A_{2,1}, \ A_{2,2}$ should be a solutions to the following system of equations
\begin{equation}\label{alb_5}
\begin{cases}
\frac{1}{30}A_{2,2} &=1, \\
A_{2,1} &=1, \\
30A_{2,0}-A_{2,2}&=0.
\end{cases}
\end{equation}
The solutions to the system (\hyperref[alb_5]{2.14}) are following: $A_{2,2}=30, \ A_{2,1}=0, \ A_{2,0}=1$. Therefore, the polynomial $g_2(T,k)=\sum_{j=0}^2 A_{2,j}k^j(T-k)^j$ takes the form
\begin{equation}\label{alb_6}
g_2(T,k)=\sum_{j=0}^2 A_{2,j}k^j(T-k)^j=30k^2(T-k)^2+1.
\end{equation}
Let be $\ell=T\in\mathbb{N}$, therefore
\begin{equation}
T^5=\sum_{k=1}^{T}30k^2(T-k)^2+1=P_2(T,T),
\end{equation}
see (\hyperref[f1]{1.1}) for $P_2(T,T)$. To show that symmetry (\hyperref[symmetry_2]{1.4}) holds for (\hyperref[alb_6]{2.15}), let's construct a triangular array filled by the values of $g_2(T,k)=30k^2(T-k)^2+1$, where $T\geq 0, \ T\in\mathbb{N}$ is row and $k$ shows the $k$-th term of the $T-$th row, see (\hyperref[symmetry_1]{1.3}) for definition of $g_m(T,k)$
\begin{table}[H]
\begin{tabular}{rccccccccccc}
$T=0$&  &    &    &    &    &  1 &    &    &    &    &   \\\noalign{\smallskip\smallskip}
$T=1$&  &    &    &    &  1 &    &  1 &    &    &    &   \\\noalign{\smallskip\smallskip}
$T=2$&  &    &    &  1 &    &  31 &    &  1 &    &    &   \\\noalign{\smallskip\smallskip}
$T=3$&  &    &  1 &    &  121 &    &  121 &    &  1 &    &   \\\noalign{\smallskip\smallskip}
$T=4$&  &  1 &    &  271 &    &  481 &    &  271 &    &  1 &   \\\noalign{\smallskip\smallskip}
$T=5$&1 &    &  481 &    & 1081 &    & 1081 &    &  481 &    & 1 \\\noalign{\smallskip\smallskip}
\end{tabular}
\caption{Triangle generated by $g_2(T,k)=30k^2(T-k)^2+1, \ 0\leq T \leq 5, \ 0\leq k \leq T$,
sequence \urlNewWindow{https://oeis.org/A300656} in OEIS, \cite{8}.} \label{fig_2}
\end{table}
The sum of the $T$-th row terms of \hyperref[fig_2]{Table (2)} starting from $k=1$ gives the partial case $P_2(T,T)=T^5$ of (\hyperref[f1]{1.1}). Below we show initial ten polynomials $P_m(\ell,T)=\sum_{k=0}^{m}(-1)^{m-k}U_m(\ell,k)\cdot T^k$, for $m=2$:
\begin{small}
\begin{equation}\label{u_2}
\begin{split}
&P_2(\ell, T)=\sum_{k=1}^{\ell}\sum_{j=0}^2 A_{2,j}k^j(T-k)^j=\sum\limits_{k=1}^{\ell}30k^2(T-k)^2+1=
\sum_{k=0}^{2}(-1)^{2-k}U_2(\ell,k)\cdot T^k \\
&=10 \ell^3 + 15 \ell^4 + 6 \ell^5 - 15 \ell^2 T - 30 \ell^3 T - 15 \ell^4 T + 5 \ell T^2 +
 15 \ell^2 T^2 + 10 \ell^3 T^2\\
&=\begin{cases}
\ell=1: 31 - 60 T + 30 T^2 &=U_2(1,0)\cdot T^0-U_2(1,1)\cdot T^1+U_2(1,2)\cdot T^2=P_2(1,T) \\
\ell=2: 512 - 540 T + 150 T^2 &=U_2(2,0)\cdot T^0-U_2(2,1)\cdot T^1+U_2(2,2)\cdot T^2=P_2(2,T)\\
\ell=3: 2943 - 2160 T + 420 T^2 &=U_2(3,0)\cdot T^0-U_2(3,1)\cdot T^1+U_2(3,2)\cdot T^2=P_2(3,T)\\
\ell=4: 10624 - 6000 T + 900 T^2 &=U_2(4,0)\cdot T^0-U_2(4,1)\cdot T^1+U_2(4,2)\cdot T^2=P_2(4,T)\\
\ell=5: 29375 - 13500 T + 1650 T^2 &=U_2(5,0)\cdot T^0-U_2(5,1)\cdot T^1+U_2(5,2)\cdot T^2=P_2(5,T)\\
\ell=6: 68256 - 26460 T + 2730 T^2 &=U_2(6,0)\cdot T^0-U_2(6,1)\cdot T^1+U_2(6,2)\cdot T^2=P_2(6,T)\\
\ell=7: 140287 - 47040 T + 4200 T^2 &=U_2(7,0)\cdot T^0-U_2(7,1)\cdot T^1+U_2(7,2)\cdot T^2=P_2(7,T)\\
\ell=8: 263168 - 77760 T + 6120 T^2 &=U_2(8,0)\cdot T^0-U_2(8,1)\cdot T^1+U_2(8,2)\cdot T^2=P_2(8,T)\\
\ell=9: 459999 - 121500 T + 8550 T^2 &=U_2(9,0)\cdot T^0-U_2(9,1)\cdot T^1+U_2(9,2)\cdot T^2=P_2(9,T)\\
\ell=10: 760000 - 181500 T + 11550 T^2 &=U_2(10,0)\cdot T^0-U_2(10,1)\cdot T^1+U_2(10,2)\cdot T^2=P_2(10,T)\\
\end{cases}\\
&=T^5 \ \mathrm{as} \ \ell = T\in\mathbb{N}.
\end{split}
\end{equation}
\end{small}
The coefficients $U_2(\ell,k), \ 0\leq k\leq 2$ in (\hyperref[u_2]{2.17}) are terms of the sequence \urlNewWindow{https://oeis.org/A316349}, \cite{17}. The following figure shows graphically the intersections of the polynomials (\hyperref[u_2]{2.17}) with monomial $T^5$ for $\ell\leq3$:
\begin{figure}[H]
\centering
\includegraphics[width=400px, keepaspectratio]{figure_n_5.eps}
\captionof{figure}{Intersections of polynomials $P_2(\ell_1=1,T), \ P_2(\ell_2=2,T), \ P_2(\ell_3=3,T)$ and monomial $T^5$ in points $\ell_1=1, \ \ell_2=2, \ \ell_3=3$ according to the identity (\hyperref[main_identity]{1.2}).}
\end{figure}
\subsection{Example for \texorpdfstring{$m=3$}{m=3}}
Consider the polynomial $P_3(\ell,T)=\sum_{k=1}^{\ell}\sum_{j=0}^3 A_{3,j}k^j(T-k)^j$.
The coefficients $A_{3,0}, \ A_{3,1}, \ A_{3,2}, \ A_{3,3}$ in $P_3(\ell,T)$ are following: $A_{3,0}=1, \ A_{3,1}=0, \ A_{3,2}=-14, \ A_{3,3}=140$. Thus, $g_3(T,k)$ is
\begin{equation}\label{7th_pow_generating_function}
g_3(T,k)=\sum_{j=0}^3 A_{3,j}k^j(T-k)^j=140k^3(T-k)^3-14k^2(T-k)^2+1.
\end{equation}
Let be $\ell=T\in\mathbb{N}$, therefore
\begin{equation}\label{seventh_power_identity}
T^7=\sum_{k=1}^{T} 140k^3(T-k)^3-14k^2(T-k)^2+1=P_3(T,T),
\end{equation}
see (\hyperref[f1]{1.1}) for $P_3(T,T)$. To show that symmetry (\hyperref[symmetry_2]{1.4}) holds for (\hyperref[7th_pow_generating_function]{2.17}), let's construct a triangular array filled by the values of $g_3(T,k)$, where $T$ is row and $k$ shows the $k$-th term of the $T-$th row, see (\hyperref[symmetry_1]{1.3}) for definition of $g_m(T,k)$

\begin{table}[H]
\begin{tabular}{rccccccccccc}
$T=0$&  &    &    &    &    &  1 &    &    &    &    &   \\\noalign{\smallskip\smallskip}
$T=1$&  &    &    &    &  1 &    &  1 &    &    &    &   \\\noalign{\smallskip\smallskip}
$T=2$&  &    &    &  1 &    &  127 &    &  1 &    &    &   \\\noalign{\smallskip\smallskip}
$T=3$&  &    &  1 &    &  1093 &    &  1093 &    &  1 &    &   \\\noalign{\smallskip\smallskip}
$T=4$&  &  1 &    &  3793 &    &  8905 &    &  3793 &    &  1 &   \\\noalign{\smallskip\smallskip}
$T=5$&1 &    &  8905 &    & 30157 &    & 30157 &    &  8905 &    & 1 \\\noalign{\smallskip\smallskip}
\end{tabular}
\caption{Triangle generated by $g_3(T,k)=140k^3(T-k)^3-14k^2(T-k)^2+1, \ 0\leq T\leq 5, \ 0\leq k \leq T$,
sequence \urlNewWindow{https://oeis.org/A300785} in OEIS, \cite{9}.} \label{fig_3}
\end{table}
The sum of the $T$-th row terms of \hyperref[fig_3]{Table (3)} starting from $k=1$ gives the partial case $P_3(T,T)=T^7$ of (\hyperref[f1]{1.1}).
Below we show initial ten polynomials $P_m(\ell,T)=\sum_{k=0}^{m}(-1)^{m-k}U_m(\ell,k)\cdot T^k$, for $m=3$:
\begin{small}
\begin{equation}\label{u_3}
\begin{split}
P_3(\ell,T)
&=\sum_{k=1}^{\ell}\sum_{j=0}^3 A_{3,j}k^j(T-k)^j=\sum\limits_{1\leq k \leq \ell}140k^3(T-k)^3-14k^2(T-k)^2+1=\sum_{k=0}^{3}(-1)^{3-k}U_3(\ell,k)\cdot T^k\\
&=7 \ell^2 + 28 \ell^3 - 70 \ell^5 - 70 \ell^6 - 20 \ell^7 - 7 \ell T - 42 \ell^2 T +
 175 \ell^4 T + 210 \ell^5 T + 70 \ell^6 T + 14 \ell T^2 - 140 \ell^3 T^2 \\
&-210 \ell^4 T^2 - 84 \ell^5 T^2 + 35 \ell^2 T^3 + 70 \ell^3 T^3 + 35 \ell^4 T^3 \\
&=\begin{cases}
\ell=1 :&  -125 + 406 T - 420 T^2 + 140 T^3=P_3(1,T)\\
\ell=2 :&  -9028 + 13818 T - 7140 T^2 + 1260 T^3=P_3(2,T)\\
\ell=3 :&  -110961 + 115836 T - 41160 T^2 + 5040 T^3=P_3(3,T)\\
\ell=4 :&  -684176 + 545860 T - 148680 T^2 + 14000 T^3=P_3(4,T)\\
\ell=5 :&  -2871325 + 1858290 T - 411180 T^2 + 31500 T^3=P_3(5,T)\\
\ell=6 :&  -9402660 + 5124126 T - 955500 T^2 + 61740 T^3=P_3(6,T)\\
\ell=7 :&  -25872833 + 12182968 T - 1963920 T^2 + 109760 T^3=P_3(7,T)\\
\ell=8 :&  -62572096 + 25945416 T - 3684240 T^2 + 181440 T^3=P_3(8,T)\\
\ell=9 :&  -136972701 + 50745870 T - 6439860 T^2 + 283500 T^3=P_3(9,T)\\
\ell=10 :& -276971300 + 92745730 T - 10639860 T^2 + 423500 T^3=P_3(10,T)
\end{cases} \\
&=T^7 \ \mathrm{as} \ \ell = T\in\mathbb{N}.
\end{split}
\end{equation}
\end{small}

The coefficients $U_3(T,k), \ 0\leq k\leq 3$ in (\hyperref[u_3]{2.20}) are terms of the sequence \urlNewWindow{https://oeis.org/A316387}, \cite{18}. The following figure shows graphically the intersections of the polynomials (\hyperref[u_3]{2.20}) with monomial $T^7$ for $\ell\leq3$:
\begin{figure}[H]
\centering
\includegraphics[width=450px, keepaspectratio]{figure_n_7.eps}
\captionof{figure}{Intersections of polynomials $P_3(\ell_1=1,T), \ P_3(\ell_2=2,T), \ P_3(\ell_3=3,T), \ P_3(\ell_4=4,T)$ and monomial $T^7$ in points $\ell_1=1, \ \ell_2=2, \ \ell_3=3, \ \ell_4=4$ according to the identity (\hyperref[main_identity]{1.2}).}
\end{figure}
Reviewing the \hyperref[fig_1]{Table (1)}, \hyperref[fig_2]{Table (2)}, \hyperref[fig_3]{Table (3)} we can observe the following relation between $g_m(n,k)$ and Pascal's identity, \cite{30}
\begin{ppty} (Relation between $g_m(n,k)$ and Pascal's identity.)
\begin{equation*}
\begin{split}
(n+1)^{2m+1}-1
&=\sum_{s=1}^{n}[(s+1)^{2m+1}-s^{2m+1}]=\sum_{p=0}^{2m}\binom{k+1}{p}(1^p+2^p+\cdots+n^p)\\
&=\sum_{k=1}^{n-1}g_m(n,k),
\end{split}
\end{equation*}
\end{ppty}
see (\hyperref[symmetry_1]{1.3}) for definition of $g_m(T,k)$.
\begin{ppty} (Linear recurrence of $g_m(T,k)$ be means of Binomial Transform.)
Review the function $g_m(T,k)$, which is defined by $m$-order polynomial, see (\hyperref[symmetry_1]{1.3}). It's well known fact that the finite difference $\Delta^u P_m(n)$ of m-order polynomial $P_m(n)$ is equal to zero for $u\geq m+1$, see \cite{31}. Recall Binomial transform of sequence $a_n$. D.E. Knuth has introduced the binomial transform in \cite{32} as follows
\begin{equation*}
\widehat{a}_n = \sum_{k=0}^{n} \binom{n}{k} (-1)^k a_k.
\end{equation*}
In particular, $\widehat{a}_n=\Delta^n a_n$, therefore, for $g_m(T,k)$ we have
\begin{equation*}
\widehat{g_m(T,k)}_n = \sum_{s=0}^{n} (-1)^{s+1} \binom{n}{s} g_m(T+(n-1)-j,k) = 0, \ \mathrm{for} \ n\geq m+1,
\end{equation*}
Let be $n=m+1$, thus,
\begin{equation*}
\widehat{g_m(T,k)}_{m+1} = \sum_{s=0}^{n} (-1)^{s+1} \binom{m+1}{s} g_m(T+m-s,k) = 0,
\end{equation*}
therefore, it gives us opportunity to represent $g_m(T+m-r,k)$ recursively for every $0 \leq r\leq m$ as follows
\begin{equation*}
(-1)^{r+1} \binom{m+1}{r} g_m(T+m-r,k)=\sum_{s\in\mathbb{N}_0/\{r\}} (-1)^{s+1} \binom{m+1}{s} g_m(T+m-s,k).
\end{equation*}
In particular,
\begin{equation*}
g_m(T+n,k) = \sum_{s=1}^{n} (-1)^{s+1} \binom{n}{s} g_m(T+(n-1)-j,k), \ n\geq m+1.
\end{equation*}
\end{ppty}
\begin{ppty} (Property of polynomial $P_m(\ell,T)$.)
For every $\ell=T+1$ in $P_m(\ell,T)$ holds
\begin{equation*}
P_m(T+1,T)=T^{2m+1}-1.
\end{equation*}
\end{ppty}
\subsection{Derivation of coefficients \texorpdfstring{$A_{m,j}$}{A(m,j)}}
To reach a recurrent formula of $A_{m,j}, \ m\geq0$, first let's fix the unused values of $A_{m,j}=0, \ \mathrm{for} \ j<0 \ \mathrm{or} \ j>m$, so we don't need to consider the summation range for $j$. By the symmetry (\hyperref[symmetry_2]{1.4}) we can rewrite (\hyperref[f1]{1.1}) as follows
\begin{equation}\label{gen_5_2}
P_m(\ell,T)=\sum_{k=0}^{\ell-1}\sum_{j=0}^m A_{m,j}k^j(T-k)^j
\end{equation}
By expanding $(T-k)^j$ in r.h.s of (\hyperref[gen_5_2]{2.21}) and using Faulhaber's formula \cite{11}, the result is

\begin{equation}\label{gen_5}
\begin{split}
&\sum_{k=0}^{\ell-1} (T-k)^j k^j =\sum_{k=0}^{\ell-1} \sum_{i} \binom{j}i T^{j-i} (-1)^i k^{i+j}\\
&=\sum_{i} \binom{j}{i} T^{j-i} \frac{(-1)^i}{i+j+1} \left[ \sum_{t} \binom{i+j+1}t B_t \ell^{i+j+1-t} - B_{i+j+1}\right] \\
&=\underbrace{\sum_{i,t}\binom{j}{i} \frac{(-1)^i}{i+j+1} \binom{i+j+1}t B_tT^{j-i}\ell^{i+j+1-t}}_{(\star)}
- \underbrace{\sum_{i} \binom{j}{i}  \frac{(-1)^i}{i+j+1} B_{i+j+1} T^{j-i}}_{(\diamond)}
\end{split}
\end{equation}
where $B_t$ are Bernoulli numbers \cite{14}. Now, we notice that
\begin{equation}\label{gen_6}
\sum_{i} \binom{j}{i} \frac{(-1)^i}{i+j+1} \binom{i+j+1}t
=\begin{cases}
\frac{1}{(2j+1)\binom{2j}j}, & \text{if } t=0;\\
\frac{(-1)^j}{t}\binom{j}{2j-t+1}, & \text{if } t>0.
\end{cases}
\end{equation}
In particular, the last sum is zero for $0<t\leq j$. Now, by substituting to the $(\star)$ part of
(\hyperref[gen_5]{2.22}) the result of (\hyperref[gen_6]{2.23}), we have
\begin{equation}\label{gen_6_1_2}
\begin{split}
\sum_{i,t}\binom{j}{i} \frac{(-1)^i}{i+j+1} \binom{i+j+1}{t} B_tT^{j-i}\ell^{i+j+1-t}
&= \frac{1}{(2j+1)\binom{2j}j}T^{j}\ell^{j+1}\\
&+\sum_{t>0} \frac{(-1)^j}{t}\binom{j}{2j-t+1} B_t T^{j-i}\ell^{i+j+1-t}
\end{split}
\end{equation}
By means of (\hyperref[gen_6_1_2]{2.24}), expression (\hyperref[gen_5]{2.22}) takes the form
\begin{equation}\label{gen_6_1}
\begin{split}
\sum_{k=0}^{\ell-1} (T-k)^j k^j
&=\underbrace{\frac{1}{(2j+1)\binom{2j}j}T^{j}\ell^{j+1}+\sum_{t>0}\frac{(-1)^j}{t}\binom{j}{2j-t+1} B_t T^{j-i}\ell^{i+j+1-t}}_{(\star)} \\
&- \underbrace{\sum_{i} \binom{j}{i}  \frac{(-1)^i}{i+j+1} B_{i+j+1} T^{j-i}}_{(\diamond)}
\end{split}
\end{equation}
We have to remember that if the sum over some variable $i$ contains $\binom{j}{i}$, then instead of limiting its summation range to $i\in[0, \ j]$, we can let $i\in[-\infty, \ +\infty]$ since $\binom{j}{i} = 0$ for $i$ outside the range $i\in[0, \ j]$ (i.e., when $i<0$ or $i>j$). It's much easier to review such sum as sum from $-\infty$ to $+\infty$ (unless specified otherwise), where only a finite number of terms are nonzero, this fact is also discussed in \cite{12}. Therefore, we haven't shown detailed bounds of summation in above derivation.
Now, we keep our attention on (\hyperref[gen_6_1]{2.25}).
To combine or cancel identical terms across the two sums in (\hyperref[gen_6_1]{2.25}) more easily, let introduce $\kappa=2j+1-t$ to $(\star)$ and $\kappa=j-i$ to $(\diamond)$, respectively
\begin{equation}\label{gen_7}
\begin{split}
\sum_{k=0}^{\ell-1} (T-k)^j k^j
&=\frac{1}{(2j+1)\binom{2j}j} T^{j}\ell^{j+1}
+\sum_{\kappa} \frac{(-1)^j}{2j+1-\kappa}\binom{j}{\kappa}B_{2j+1-\kappa}T^{j-i}\ell^{\kappa+i-j} \\
&-\sum_{\kappa} \binom{j}{\kappa} \frac{(-1)^{j-\kappa}}{2j+1-\kappa} B_{2j+1-\kappa}T^{\kappa}
\end{split}
\end{equation}
Let be $\ell=T$ in (\hyperref[gen_7]{2.26}), thus
\begin{equation}\label{gen_7_1}
\begin{split}
\sum_{k=0}^{T-1} (T-k)^j k^j
&=\frac{1}{(2j+1)\binom{2j}j} T^{2j+1}
+\sum_{\kappa} \frac{(-1)^j}{2j+1-\kappa}\binom{j}{\kappa}B_{2j+1-\kappa}T^{\kappa} \\
&-\sum_{\kappa} \binom{j}{\kappa} \frac{(-1)^{j-\kappa}}{2j+1-\kappa} B_{2j+1-\kappa}T^{\kappa}\\
&=\frac{1}{(2j+1)\binom{2j}j} T^{2j+1} + 2\sum_{\text{odd }\kappa}
\frac{(-1)^j}{2j+1-\kappa}\binom{j}{\kappa}B_{2j+1-\kappa}T^{\kappa}.
\end{split}
\end{equation}
Now, using the definition of $A_{m,j}$,
\begin{equation*}
T^{2m+1}=\sum_{k=0}^{T-1}\sum_{j=0}^{m} A_{m,j}k^j(T-k)^j,
\end{equation*}
From (\hyperref[gen_7_1]{2.27}) we can obtain the following identity for polynomials in $T$
\begin{equation}\label{gen_8}
\begin{split}
&\sum_{j}A_{m,j}\frac{1}{(2j+1)\binom{2j}j}T^{2j+1}
+ 2\sum_{j, \text{ odd }\kappa} A_{m,j} \binom{j}{\kappa} \frac{(-1)^j}{2j+1-\kappa} B_{2j+1-\kappa}T^{\kappa}\\
&\equiv n^{2m+1}.
\end{split}
\end{equation}
Taking the coefficient of $n^{2j+1}$ in the above expression, we get $A_{m,m} = (2m+1) \binom{2m}{m},$ and taking the
coefficient of $x^{2d+1}$ for an integer $d$ in the range $m/2 \leq d < m$ we get $A_{m,d}=0$.
Taking the coefficient of $n^{2d+1}$ in (\hyperref[gen_8]{2.28}) for $m/4 \leq d < m/2$ , we get
\begin{equation*}\label{gen_10}
A_{m,d} \frac{1}{(2d+1)\binom{2d}{d}} + 2 (2m+1) \binom{2m}{m} \binom{m}{2d+1} \frac{(-1)^m}{2m-2d} B_{2m-2d} = 0,
\end{equation*}
i.e
\begin{equation*}\label{gen_11}
A_{m,d} = (-1)^{m-1} \frac{(2m+1)!}{d!d!m!(m-2d-1)!}\frac{1}{m-d} B_{2m-2d}.
\end{equation*}
Continuing similarly, we can express $A_{m,j}$ for each integer $j$ in the range $m/2^{s+1}\leq j< m/2^s$ (iterating consecutively $s=1,2,...$) via the previously determined values of $A_{m,d}, \ d<j$ as follows
\begin{equation*}\label{gen_12}
A_{m,j} = (2j+1)\binom{2j}{j} \sum_{d=2j+1}^{m} A_{m,d} \binom{d}{2j+1} \frac{(-1)^{d-1}}{d-j} B_{2d-2j}.
\end{equation*}
The same formula holds also for $m=0$. Note that the $m$ in above sum must satisfy $m\geq2j+1$ to return a nonzero term $A_{m,j}$.
\begin{defn}\label{gen_13}
\textit{(Definition of $A_{m,j}$ coefficients)}
\begin{equation*}
A_{m,j}:=
\begin{cases}
0, & \mathrm{if } \ j<0 \ \mathrm{or } \ j>m, \\
(2j+1)\binom{2j}{j} \sum_{d=2j+1}^{m} A_{m,d} \binom{d}{2j+1} \frac{(-1)^{d-1}}{d-j} B_{2d-2j}, & \mathrm{if } \ 0 \leq j < m, \\
(2j+1)\binom{2j}{j}, & \mathrm{if } \ j=m. \\
\end{cases}
\end{equation*}
\end{defn}
Five initial rows of the triangle generated by $A_{m,j}, \ m\geq0, \ 0\leq j \leq m$ are

\begin{table}[H]
\begin{tabular}{rccccccccccc}
$m=0$&  &    &    &    &    &  1 &    &    &    &    &   \\\noalign{\smallskip\smallskip}
$m=1$&  &    &    &    &  1 &    &  6 &    &    &    &   \\\noalign{\smallskip\smallskip}
$m=2$&  &    &    &  1 &    &  0 &    &  30 &    &    &   \\\noalign{\smallskip\smallskip}
$m=3$&  &    &  1 &    &  -14 &    &  0 &    &  140 &    &   \\\noalign{\smallskip\smallskip}
$m=4$&  &  1 &    &  -120 &    &  0 &    &  0 &    &  630 &   \\\noalign{\smallskip\smallskip}
$m=5$&1 &    &  -1386 &    & 660 &    & 0 &    &  0 &    & 2772 \\\noalign{\smallskip\smallskip}
\end{tabular}
\caption{Triangle generated by $A_{m,j}, \ 0\leq j \leq m, \ m\geq0$, see definition (\hyperref[gen_13]{2.29}).} \label{fig_4}
\end{table}

Zeroes appear in \hyperref[fig_4]{Table (4)} for $m\geq2$, with respect to regularity:
\begin{equation*}
\begin{cases}
A_{m,j} = 0, \ &\mathrm{if} \ j=k+\left\lfloor\frac{m-1}{2}\right\rfloor, \ 1\leq k\leq \left\lceil\frac{m-1}{2}\right\rceil,\\
A_{m,j} \neq 0, \ &\mathrm{otherwise},
\end{cases}
\end{equation*}
see \urlNewWindow{https://oeis.org/A138099}.

Note that starting from row $m\geq11$ the terms of the \hyperref[fig_4]{Table (4)} consists the fractional numbers, for example, $A_{11,1}=800361655623.6$. One can find a complete list of the numerators and denominators of $A_{m,j}$ in OEIS under the identifiers  \urlNewWindow{https://oeis.org/A302971} and \urlNewWindow{https://oeis.org/A304042}, respectively, see \cite{17},\cite{18}. The terms $A_{m,m}$ are \urlNewWindow{https://oeis.org/A002457}. Below we show a few properties of coefficients $A_{m,j}$.
\begin{ppty} (Summation of $A_{m,j}$.)
\begin{equation*}
\sum_{j=0}^{m} A_{m,j} \ = \ 2^{2m+1}-1.
\end{equation*}
\end{ppty}
\begin{ppty} (Odd power identity be means of certain integrals.)
\begin{equation}
T^{2m+1}=\int\limits_{0}^{T}A_{m,m}k^m(T-k)^mdk=A_{m,m}\int\limits_{0}^{T}k^m(T-k)^mdk.
\end{equation}
\begin{proof}
\begin{equation}
\begin{split}
T^{2m+1}
&=\sum_{s=0}^{m}(-1)^s A_{m,m} \binom{m}{s} T^{m-s} \int\limits_{0}^{T}k^{m+s}dk\\
&=\sum_{s=0}^{m}(-1)^s A_{m,m} \binom{m}{s} T^{m-s}\frac{T^{m+s+1}}{m+s+1}\\
&=T^{2m+1}\sum_{s=0}^{m}\ddfrac{(-1)^s A_{m,m}}{m+s+1} \binom{m}{s}.
\end{split}
\end{equation}
\end{proof}
\end{ppty}
It implies that
\begin{equation*}
\sum_{s=0}^{m}\ddfrac{(-1)^s A_{m,m}}{m+s+1} \binom{m}{s} = 1,
\end{equation*}
Therefore,
\begin{equation*}
A_{m,m} = \ddfrac{1}{\sum_{s=0}^{m}\frac{(-1)^s}{m+s+1}\binom{m}{s}}.
\end{equation*}
\subsection{Derivation of coefficients \texorpdfstring{$U_m(\ell,k)$}{Um(l,k)}}
To find a recurrence for coefficients $U_m(\ell,k)$ for every $m\geq1$ let's express the polynomials (\hyperref[f1]{1.1}) in terms of $A_{m,j}$ and Bernoulli numbers. To do so, let's expand the binomial
$(T-k)^j$ in the l.h.s. of (\hyperref[f1]{1.1}) and change of the order of summation:
\begin{equation}\label{sum_2}
\begin{split}
\sum_{k=1}^{\ell}\sum_{j=0}^m A_{m,j}k^j(T-k)^j &= \sum_{k=1}^{\ell}\sum_{j=0}^m A_{m,j}k^j\sum_{t=0}^j(-1)^{j-t}\binom{j}{t}T^t k^{j-t}\\
&=\sum_{t=0}^m T^t \sum_{k=1}^{\ell}\sum_{j=t}^m (-1)^{j-t}\binom{j}{t}A_{m,j}k^{2j-t}.
\end{split}
\end{equation}
Now, taking the coefficient of $T^t$ in (\hyperref[sum_2]{2.30}) gives:
\begin{equation*}
U_m(\ell,t) = (-1)^m \sum_{k=1}^{\ell}\sum_{j=t}^m (-1)^j\binom{j}{t}A_{m,j}k^{2j-t}.
\end{equation*}
From this formula it may be not immediately clear why $U_m(\ell,t)$ represent polynomials in $\ell$. However, this can be seen if we change the summation order again and use Faulhaber's formula to obtain:
\begin{equation}\label{maxal1}
U_m(\ell,t) = (-1)^m \sum_{j=t}^m \binom{j}{t}A_{m,j} \frac{(-1)^j}{2j-t+1}\sum_{l=0}^{2j-t} \binom{2j-t+1}{l}B_{l}\ell^{2j-t+1-l}.
\end{equation}
Introducing $\kappa=2j-t+1-l$ to (\hyperref[maxal1]{2.31}) we further get the formula:
\begin{equation*}
U_m(\ell,t) = (-1)^m \sum_{\kappa=1}^{2m-t+1} \ell^\kappa \sum_{j=t}^m \binom{j}{t}A_{m,j} \frac{(-1)^j}{2j-t+1}\binom{2j-t+1}{\kappa}B_{2j-t+1-\kappa},
\end{equation*}
which allows easily compute the coefficient of $\ell^{\kappa}$ in $U_m(\ell,t)$ for each $\kappa$. In above formulae we assume that $B_1=+\frac12$.
\section{Acknowledgements}
We would like to thank to Dr. Max Alekseyev (Department of Mathematics and Computational Biology,
George Washington University) for sufficient help in the derivation of $A_{m,j}$ coefficients,
Dr. Hansruedi Widmer for his useful comments concerning the derivation of coefficients $A_{m,j}$,
Dr. Ron Knott (Visiting Fellow, Dept. of Mathematics at University of Surrey) for his useful
suggestions on the writing of this article, and to Mr. Albert Tkaczyk for his help in derivation
of the cases $m=2$ and $m=3$.
Also, we'd like to thank to OEIS editors Michel Marcus, Peter Luschny, Jon E. Schoenfield and others
for their patient, faithful volunteer work and for useful comments and suggestions during
the editing of sequences, concerned with this manuscript. We, also, greatly appreciate the
preliminary proofread and suggestions by
\urlNewWindow{http://www.cantab.net/users/bnr/}.
\section{Conclusion}
In this manuscript we have derived and discussed the polynomials $P_m(\ell,T)$, such that equal to the monomial $T^{2m+1}$ as $\ell=T\in\mathbb{N}$.
\section{Supplementary Files}
We provide the following supplementary files to support our study:
\begin{itemize}
  \item \urlNewWindow{https://kolosovpetro.github.io/arxiv_1603_02468/identity_1_1.txt} - Mathematica program, implementation the left-hand side of the identity (\hyperref[f1]{1.1}), that is $T^{2m+1}=\sum_{k=1}^{\ell}\sum_{j=0}^m A_{m,j}k^j(T-k)^j, \ \ell=T\in\mathbb{N}.$
  \item \urlNewWindow{https://kolosovpetro.github.io/arxiv_1603_02468/identity_1_1_r_h_s.txt} - Mathematica program, implementation the right-hand side of the identity (\hyperref[f1]{1.1}), that is $T^{2m+1}=\sum_{k=0}^{m}(-1)^{m-k}U_m(\ell,k)\cdot T^k, \ \ell=T\in\mathbb{N}.$
  \item \urlNewWindow{https://kolosovpetro.github.io/arxiv_1603_02468/u_coefficients_in_row.txt} - Mathematica program, that lists the coefficients $U_m(\ell,k)$ in (\hyperref[f1]{1.1}) for given $m$. (By default $m=2$)
  \item \urlNewWindow{https://kolosovpetro.github.io/pdf/error_of_approximation.pdf} - pdf file, shows the error of approximation of the odd power $T^{2m+1}$ by (\hyperref[f1]{1.1}).
\end{itemize}
\begin{thebibliography}{9}
\bibitem{1}
        Donald E. Knuth., Johann Faulhaber and Sums of Powers, pp. 9-10., arXiv preprint,
\urlNewWindow{https://arxiv.org/abs/math/9207222v1}, 1992.
\bibitem{2}
        John Riordan, Combinatorial Identities (New York: John Wiley \& Sons, 1968).
\bibitem{3}
        OEIS Foundation Inc. (2018), The On-Line Encyclopedia of Integer Sequences \urlNewWindow{https://oeis.org/A287326}.
\bibitem{4}
        Weisstein, Eric W. "\href{http://mathworld.wolfram.com/FiniteDifference.html}{Finite Difference.}" From \href{http://mathworld.wolfram.com/}{Mathworld}.
\bibitem{5}
        OEIS Foundation Inc. (2018), The On-Line Encyclopedia of Integer Sequences, \urlNewWindow{https://oeis.org/A028896}.
\bibitem{6}
        OEIS Foundation Inc. (2018), The On-Line Encyclopedia of Integer Sequences, \urlNewWindow{https://oeis.org/A275709}.
\bibitem{7}
        Weisstein, Eric W. "\href{http://mathworld.wolfram.com/BernoulliNumber.html}{Bernoulli Number.}" From \href{http://mathworld.wolfram.com/}{Mathworld}--A Wolfram Web Resource.
\bibitem{8}
        OEIS Foundation Inc. (2018), The On-Line Encyclopedia of Integer Sequences, \urlNewWindow{https://oeis.org/A300656}.
\bibitem{9}
        OEIS Foundation Inc. (2018), The On-Line Encyclopedia of Integer Sequences, \urlNewWindow{https://oeis.org/A300785}.
\bibitem{10}
        Ronald Graham, Donald Knuth, and Oren Patashnik (1989-01-05). "Binomial coefficients". Concrete Mathematics (1st ed.). Addison Wesley Longman Publishing Co. p. 162. ISBN 0-201-14236-8, \href{https://smarturl.it/concrete_mathematics}{online copy}.
\bibitem{11}
         Johann Faulhaber, {\sl Academia Algebr{\ae}}, Darinnen die miraculosische Inventiones zuden h\"ochsten Cossen weiters {\it continuirt\/} und {\it profitiert\/} werden. Augspurg, bey Johann Ulrich Sch\"onigs, 1631.
         (Call number {\tt QA154.8{\kern3pt}F3{\kern3pt}1631a{\kern3pt}f{\kern3pt}MATH} at Stanford University Libraries.),  \href{http://digital.slub-dresden.de/fileadmin/data/272635758/272635758_tif/jpegs/272635758.pdf}{online copy}.
\bibitem{12}
        Donald E. Knuth., Two notes on notation., pp. 1-2, arXiv preprint,
        \urlNewWindow{https://arxiv.org/abs/math/9205211}, 1992.
\bibitem{13}
        OEIS Foundation Inc. (2018), The On-Line Encyclopedia of Integer Sequences, \urlNewWindow{https://oeis.org/A302971}.
\bibitem{14}
        OEIS Foundation Inc. (2018), The On-Line Encyclopedia of Integer Sequences, \urlNewWindow{https://oeis.org/A304042}.
\bibitem{15}
        OEIS Foundation Inc. (2018), The On-Line Encyclopedia of Integer Sequences, \urlNewWindow{https://oeis.org/A002457}.
\bibitem{16}
        Donald E. Knuth, The Art of Computer Programming Vol. 3, (1973) Addison-Wesley, Reading, MA.
\bibitem{17}
        OEIS Foundation Inc. (2018), The On-Line Encyclopedia of Integer Sequences, \urlNewWindow{https://oeis.org/A316349}.
\bibitem{18}
        OEIS Foundation Inc. (2018), The On-Line Encyclopedia of Integer Sequences, \urlNewWindow{https://oeis.org/A316387}.
\bibitem{22}
        John Riordan, Combinatorial Identities (New York: John Wiley \& Sons, 1968).
\bibitem{27}
        OEIS Foundation Inc. (2018), The On-Line Encyclopedia of Integer Sequences, \urlNewWindow{https://oeis.org/A303675}.
\bibitem{28}
        OEIS Foundation Inc. (2018), The On-Line Encyclopedia of Integer Sequences, \urlNewWindow{https://oeis.org/A320047}.
\bibitem{29}
         OEIS Foundation Inc. (2018), The On-Line Encyclopedia of Integer Sequences, \urlNewWindow{https://oeis.org/A000124}.
\bibitem{30}
        Kieren MacMillan, Jonathan Sondow. Proofs of power sum and binomial coefficient congruences via Pascal's identity, p. 2., arXiv preprint, \urlNewWindow{https://arxiv.org/abs/1011.0076}, 1992.
\bibitem{31}
        Bakhvalov N. S. Numerical Methods: Analysis, Algebra, Ordinary Differential Equations p. 59, 1977. (In russian)
\bibitem{32}
        Donald E. Knuth, The Art of Computer Programming Vol. 3, (1973) Addison-Wesley, Reading, MA.
\end{thebibliography}
\end{document}
